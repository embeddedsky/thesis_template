\chapter{Introduction}
\newpage 

\section{Background}
In today's interconnected and diverse world, being a good person and earning the respect of others is an essential goal for many individuals. The concept of goodness and respect is deeply rooted in various cultural, philosophical, and religious traditions. It encompasses qualities such as integrity, empathy, compassion, fairness, and ethical conduct. However, the path to becoming a good person who is respected by others is not always clear-cut, as it involves a complex interplay of personal values, societal norms, and interpersonal relationships.

\section{Motivation}
The motivation behind this thesis stems from the recognition that while there is a wealth of literature available on personal development, ethics, and interpersonal skills, there is a need for a comprehensive guide that synthesises and provides practical advice on how to cultivate goodness and earn the respect of others in today's world. This thesis aims to bridge that gap by offering a holistic framework and actionable strategies that individuals can employ to enhance their personal growth and establish meaningful connections with others.

\section{Research Objectives}
The primary objective of this research is to explore the various dimensions of being a good person and understanding the factors that contribute to gaining respect from others. Specifically, this thesis seeks to:

\begin{enumerate}
  \item Examine the theoretical foundations and philosophical perspectives on goodness and respect in different cultures and belief systems.
  \item Identify the key qualities and virtues that define a good person and contribute to earning respect from others.
  \item Investigate the influence of societal norms, cultural diversity, and personal values on perceptions of goodness and respect.
  \item Analyse the role of effective communication, empathy, and ethical decision-making in building and maintaining respectful relationships.
  \item Provide practical guidelines and strategies for individuals to develop their personal qualities, enhance their interpersonal skills, and foster respect in various contexts.
\end{enumerate}

\section{Methodology}
To achieve the research objectives, this thesis will employ a mixed-methods approach, combining both qualitative and quantitative research methods. The qualitative aspect will involve a comprehensive review of relevant literature, including philosophical texts, psychological studies, and sociological analyses. Additionally, interviews and case studies will be conducted to gather firsthand perspectives and experiences related to goodness and respect.

The quantitative aspect will entail surveys and questionnaires distributed to a diverse sample of individuals across different demographic backgrounds. This will allow for the collection of empirical data on perceptions of goodness, respect, and the factors that influence them.

\section{Thesis Structure}
The remainder of this thesis is organised as follows:

\begin{description}
  \item[Chapter 2] provides a comprehensive review of the theoretical foundations of goodness and respect in different cultural, philosophical, and religious contexts.
  \item[Chapter 3] explores the qualities and virtues that are commonly associated with being a good person and earning respect from others.
  \item[Chapter 4] examines the influence of societal norms, cultural diversity, and personal values on perceptions of goodness and respect.
  \item[Chapter 5] analyses the role of effective communication, empathy, and ethical decision-making in building and maintaining respectful relationships.
  \item[Chapter 6] presents practical guidelines and strategies for individuals to enhance their personal qualities, interpersonal skills, and foster respect in various contexts.
  \item[Chapter 7] summarises the key findings of the research, discusses their implications, and provides recommendations for further exploration in this field.
\end{description}

In conclusion, this thesis aims to contribute to the existing body of knowledge on personal development, ethics, and interpersonal relationships by offering a comprehensive guide on how to be a good person who is respected by others in the world. By exploring the theoretical foundations, examining the key qualities and virtues, investigating the influence of societal norms and personal values, and analysing the role of effective communication and empathy, this thesis seeks to provide valuable insights and practical strategies for individuals to enhance their personal growth and foster respectful relationships.

It is hoped that the findings and recommendations presented in this thesis will serve as a resource for individuals from various backgrounds who aspire to cultivate goodness and earn the respect of others. By understanding the complex dynamics involved in being a good person and navigating interpersonal relationships, individuals can make meaningful contributions to their communities and create positive change in the world.

It is important to note that this thesis does not claim to provide a definitive blueprint for goodness or a universal definition of respect. The concept of goodness and respect is subjective and can vary across cultures, belief systems, and individual perspectives. However, by integrating insights from diverse sources and engaging in a rigorous research process, this thesis aims to provide a comprehensive framework that can be adapted and personalised to individual circumstances and cultural contexts.

It is also worth mentioning that while the focus of this thesis is on personal development and interpersonal relationships, the broader implications of being a good person who is respected by others extend beyond the individual level. A society composed of individuals who embody goodness and mutual respect is more likely to promote fairness, harmony, and social cohesion. Thus, the exploration of these topics has implications for fostering a more just and compassionate world.

In summary, this introduction chapter has outlined the background, motivation, research objectives, methodology, and thesis structure for the study on how to be a good person who is respected by others in the world. By delving into the theoretical foundations, examining key qualities and virtues, investigating societal influences, and analysing the role of effective communication and empathy, this thesis aims to provide a comprehensive guide for individuals seeking personal growth and the establishment of respectful relationships. The subsequent chapters will delve into each aspect in detail, presenting a synthesis of existing knowledge and offering practical strategies for application.

\section{Publications}


\begin{table}[h]
\caption{Publications}
\label{table_publication}
\begin{center}
\small
\resizebox{\textwidth}{!}{ 
\begin{tabular}{@{}p{0.05cm}p{8.4cm}p{2.65cm}p{2.3cm}l@{}}
\toprule
\textbf{\#} &\textbf{Titile}       & \textbf{Authors} & \textbf{Jour./Conf.} & \textbf{Status} \\ \midrule
1.&\textit{Why Dashuai Wang is Good Man?} & Wang and Liu & CGPGT 2022 & Published 
 \\ 
 \hline 
2.&\textit{Why Xiaomei Liu is Good Woman?} & Wang and Liu & CGPGT 2023 & Published  \\  \hline
3.&\textit{Why Dashuai Wang and Xiaomei Liu are Good People?}  & Wang and Liu & CGPGT 2024 & Submitted \\
 \bottomrule
\end{tabular}
}
\end{center}
\end{table}